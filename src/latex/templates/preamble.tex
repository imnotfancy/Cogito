% LaTeX preamble with required packages for scientific papers with mathematical content
% This preamble is designed to be compatible with standard LaTeX distributions

% Basic packages
\usepackage[utf8]{inputenc}     % Input encoding
\usepackage[T1]{fontenc}        % Font encoding
\usepackage{lmodern}            % Latin Modern fonts
\usepackage{amsmath,amssymb}    % AMS math and symbols
\usepackage{mathtools}          % Enhanced math tools
\usepackage{graphicx}           % Graphics inclusion
\usepackage{xcolor}             % Color management

% Scientific packages
\usepackage{siunitx}            % SI units
\usepackage{booktabs}           % Professional tables
\usepackage{array}              % Enhanced tables
\usepackage{tabularx}           % Flexible tables
\usepackage{float}              % Better figure/table positioning

% Code and algorithms
\usepackage{listings}           % Code listings
\usepackage{algorithm}          % Algorithm environments
\usepackage{algorithmic}        % Algorithm content

% References and citations - Using natbib only (removed biblatex to avoid conflicts)
\usepackage[square,numbers]{natbib}

% Layout and formatting
\usepackage{geometry}           % Page layout
\usepackage{setspace}           % Line spacing
\usepackage{enumitem}           % List customization
\usepackage{caption}            % Figure and table captions

% Configure page geometry
\geometry{a4paper, margin=1in}

% Hyperlinks - Load after other packages
\usepackage{hyperref}           % Hyperlinks and PDF metadata
\hypersetup{
    colorlinks=true,            % Colored links
    linkcolor=blue,             % Color of internal links
    citecolor=blue,             % Color of citation links
    urlcolor=blue,              % Color of URL links
    pdfauthor={Critique Council},  % Default PDF metadata
    pdftitle={Scientific Methodology Analysis},
    pdfsubject={Scientific Critique},
    pdfkeywords={methodology, analysis, critique}
}

% Configure listings for code
\lstset{
    basicstyle=\small\ttfamily,
    numbers=left,
    numberstyle=\tiny,
    frame=single,
    breaklines=true,
    showstringspaces=false
}

% Custom environments and commands
\newenvironment{scientificabstract}
    {\begin{abstract}\itshape}
    {\end{abstract}}

\newcommand{\keywords}[1]{\textbf{Keywords:} #1}
\newcommand{\methodologyref}[1]{\textit{Methodology:} #1}
\newcommand{\resultssummary}[1]{\textbf{Results Summary:} #1}

% Additional math commands for scientific notation
\newcommand{\vect}[1]{\boldsymbol{#1}}
\newcommand{\deriv}[2]{\frac{\partial #1}{\partial #2}}
\newcommand{\abs}[1]{\left|#1\right|}
\newcommand{\norm}[1]{\left\|#1\right\|}
