\documentclass{article}
\usepackage[
backend=biber,
style=alphabetic,
sorting=ynt
]{biblatex}


\usepackage{geometry}
 \geometry{
 a4paper,
 total={170mm,257mm},
 left=20mm,
 top=20mm,
 }
 
 \usepackage[acronym]{glossaries}

\usepackage{optidef}
\addbibresource{mybibliography.bib}


\makeglossaries

\newglossaryentry{entryOne}
{
        name=Glossary Entry,
        description={Glossary entries are used to provide definitions for words in your document}
}    


\title{Literature Review }
\author{Author Name}
\date{November 2021}

\begin{document}

\maketitle

\section{Some Terminology}
There is none!
    
\section{Some optimization problems}
    \begin{mini}|s|[0]
    {y \in Y, u \in U}{J(y,u)}
    {} 
    {\label{eq:minimizationProblem}}{}
    \addConstraint{F(y,u)}{= 0}
    \addConstraint{c(y,u)}{ = 0}
    \addConstraint{h(y,u)}{ \ge 0}
\end{mini}

    Above, you can see a minimization problem! We can refer to \gls{entryOne} for some clues as to what this means. We can also cite how we learned about this optimization problem! \cite{terribleBook}


    
\medskip

\printglossary


\printbibliography
\end{document}
